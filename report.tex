\documentclass[degree=bachelor, tocarialchapter, pifootnote]{thuthesis}
% 选项
%   degree=[bachelor|master|doctor|postdoctor], % 必选,学位类型
%   secret,                % 可选(默认:关闭),是否有密级
%   tocarialchapter,       % 可选(默认:关闭),章目录中使用黑体(这项表示同时打开下面两项)
%   tocarialchapterentry,  % 可选(默认:关闭),单独控制章标题在目录中使用黑体
%   tocarialchapterpage,   % 可选(默认:关闭),单独控制章页码在目录中使用黑体
%   pifootnote,            % 可选(默认:关闭),页脚编号采用 pifont 字体符号,建议打开

% 所有其它可能用到的包都统一放到这里了,可以根据自己的实际添加或者删除。
\usepackage{thuthesis}

% 下面的包让argmax之类的,_{}的东西显示在argmax正下方
\usepackage{amsmath}
\usepackage{algorithmic}
\usepackage{algorithm}
\DeclareMathOperator*{\argmax}{argmax}
\DeclareMathOperator*{\argmin}{argmin}

% 定义所有的图片文件在 figures 子目录下
\graphicspath{{figures/}}

% 可以在这里修改配置文件中的定义。导言区可以使用中文。
\def\myname{王芮}

\begin{document}

%%% 封面部分å
\frontmatter
\thusetup{
  %******************************
  % 注意:
  %   1. 配置里面不要出现空行
  %   2. 不需要的配置信息可以删除
  %******************************
  %
  %=====
  % 秘级
  %=====
%  secretlevel={秘密},
%  secretyear={10},
  %
  %=========
  % 中文信息
  %=========
  %ctitle={分层强化学习算法及其在机器人仿真环境中的应用研究},
  ctitle={HAAR:基于优势函数辅助奖励的分层强化学习算法},
  cdegree={工学学士},
  cdepartment={机械工程系},
  cmajor={机械工程},
  cauthor={王芮},
  csupervisor={雷丽萍副教授},
%  ccosupervisor={某某某教授}, % 联合指导老师
  % 日期自动使用当前时间,若需指定按如下方式修改:
  % cdate={超新星纪元},
  %
  % 博士后专有部分
%  cfirstdiscipline={计算机科学与技术},
%  cseconddiscipline={系统结构},
%  postdoctordate={2009年7月——2011年7月},
%  id={编号}, % 可以留空: id={},
%  udc={UDC}, % 可以留空
%  catalognumber={分类号}, % 可以留空
  %
  %=========
  % 英文信息
  %=========
%  etitle={An Introduction to \LaTeX{} Thesis Template of Tsinghua University v\version},
  % 这块比较复杂,需要分情况讨论:
  % 1. 学术型硕士
  %    edegree:必须为Master of Arts或Master of Science(注意大小写)
  %             “哲学、文学、历史学、法学、教育学、艺术学门类,公共管理学科
  %              填写Master of Arts,其它填写Master of Science”
  %    emajor:“获得一级学科授权的学科填写一级学科名称,其它填写二级学科名称”
  % 2. 专业型硕士
  %    edegree:“填写专业学位英文名称全称”
  %    emajor:“工程硕士填写工程领域,其它专业学位不填写此项”
  % 3. 学术型博士
  %    edegree:Doctor of Philosophy(注意大小写)
  %    emajor:“获得一级学科授权的学科填写一级学科名称,其它填写二级学科名称”
  % 4. 专业型博士
  %    edegree:“填写专业学位英文名称全称”
  %    emajor:不填写此项
%  edegree={Doctor of Engineering},
%  emajor={Computer Science and Technology},
%  eauthor={Xue Ruini},
%  esupervisor={Professor Zheng Weimin},
%  eassosupervisor={Chen Wenguang},
  % 日期自动生成,若需指定按如下方式修改:
  % edate={December, 2005}
  %
  % 关键词用“英文逗号”分割
  ckeywords={强化学习, 分层强化学习, 机器人学, 规划算法, 自动驾驶, 路径规划, 优化, 策略单调性},
  ekeywords={Reinforcement Learning, Hierarchical Reinforcement Learning, Robotics, Planning Algorithms, Self-Driving, Path Planning, Optimization, Monotonic Improvement}
}

% 定义中英文摘要和关键字
\begin{cabstract}
强化学习问题中的一个重要挑战是环境的稀疏奖励问题。如果环境反馈给智能体的信号过于稀疏,智能体的学习速度会非常缓慢。为此,人们提出分层强化学习的方法,来帮助强化学习算法提高采样效率,加速学习。在分层强化学习范式当中,特别地,我们需要为下层策略设计一个辅助奖励函数,从而促进下层策略的学习。在本篇论文中,我们提出了一个分层强化学习的框架,通过基于优势函数的下层奖励函数设计,它可以单调地同时优化上下层策略。截止目前,绝大多数最前沿的分层强化学习算法都对状态空间的表达非常敏感,或者依赖于手动设计的、只在特定环境下可用的下层奖励函数。得益于我们的辅助奖励函数设计只与优势函数的值相关这一特点,我们的算法不依赖于状态空间的表示方式,因此和其他方法相比更为通用和实用,尤其是在机器人连续动作空间这类应用场景。我们从理论上证明了这种算法可以保留上下层各自优化算法的单调特性,使整体策略也得到单调优化。实验方面,在Mujoco仿真环境的多个任务中,我们的算法表现均大大优于现有的最佳算法。最后,我们还搭建了一个自动驾驶仿真环境来验证本算法在更现实和复杂的环境下的表现。我们设计了迁移实验,将算法从Mujoco实验环境迁移到了自动驾驶环岛路径规划问题中,同时也指出了在这个环境中的下一步研究方向。
\end{cabstract}

\begin{eabstract}
One of the key challenges in reinforcement learning is related to sparse reward signals from the environment. Hierarchical reinforcement learning (HRL) can be utilized to solve the credit assignment problem in a sparse reward task. In the HRL paradigms, we need to design a low-level auxiliary reward function to facilitate low-level learning. In this paper, we propose an HRL framework which monotonically  optimizes the joint policy of the upper-level and lower-level  through the design of an advantage function-based auxiliary reward. Most state-of-the-art HRL algorithms are sensitive to state representations, or require careful design of the auxiliary reward functions  for low-level training. Our method is more generic and applicable in continuous control tasks since our auxiliary reward for low-level policy update is dependent only on the advantage function of the high-level policy, therefore invariant of the state representations. We theoretically prove that monotonic improvement of the joint-policy is guaranteed in our method. Experimental results show that our algorithm performs dramatically better than other state-of-the-art methods across multiple tasks in the Mujoco learning environment. Finally, we design a complex autonomous-driving simulation environment as a testbed for our algorithm in more realistic settings. We transfer our algorithm to this new environment and report the experiment results of our framework on the self-driving car's planning task at a roundabout, as well as point out the direction for future research.

\end{eabstract}

% 如果习惯关键字跟在摘要文字后面,可以用直接命令来设置,如下:
% \ckeywords{\TeX, \LaTeX, CJK, 模板, 论文}

% \ekeywords{\TeX, \LaTeX, CJK, template, thesis}

% 如果使用授权说明扫描页,将可选参数中指定为扫描得到的 PDF 文件名,例如:
% \makecover[scan-auth.pdf]
\makecover

%% 目录
% 设置目录层级,三级小标题全部显示
\setcounter{secnumdepth}{5}
\setcounter{tocdepth}{5}
\tableofcontents

%% 符号对照表
\begin{denotation}[3cm]
\item[$s$] 状态(任意状态)
\item[$a$] 行动(任意行动)
\item[$\pi$] 策略
\item[$\pi(s)$] 策略函数
\item[$\pi(a|s)$] 策略在状态$s$下采取行动$a$的概率
\item[$r$] 奖励
\item[$r(s, a)$] 在状态$s$下采取行动$a$获得的奖励
\item[$G_t$] 时间$t$之后的总奖励
\item[$v(s)$] 状态$s$的价值函数
\item[$q_\pi(s, a)$] 在策略$\pi$下,在状态$s$下采取行动$a$的价值
\item[$V(s)$] 状态$s$的价值函数的估计值
\item[$Q(s, a)$] 状态$s$,动作$a$的价值的估计值
\item[$p(s'|s, a)$] 在状态$s$采取动作$a$,状态转移到$s'$的概率
\item[$p(s', r|s, a)$] 在状态$s$采取动作$a$,状态转移到$s'$并且获得奖励$r$的概率
\item[$\pi(a|s, \theta)$] 策略函数的参数是$\theta$,在状态$s$下采取行动$a$的概率
\item[$\pi_\theta$] 策略函数,其参数为$\theta$
\item[$\bigtriangledown \pi(a|s, \theta)$] 策略函数的参数是$\theta$,在状态$s$下采取行动$a$的概率对$\theta$的导数
\item[$\gamma$] 总奖励$G_t$计算时,对于后续$R_{t+t_i}$的折扣率
\item[$\rho$] 重要性采样比例(importance-sampling ratio)
\item[$\delta_t$] 时序差异误差(TD error)
\item[RL] 强化学习(reinforcement learning)
\item[MPC] 模型预测控制(model-predictive control)
\item[TD] 时序差异(temporal-difference)
\item[MDP] 马尔可夫过程 (Markov Decision Process)
\item[POMDP] 部分可观测的马尔可夫过程 (Partially Observable Markov Decision Process)
\item[DQN] 深度Q-学习神经网络(Deep Q Network)
\item[DRQN] 深度循环Q-学习神经网络(Deep Recurrent Q Network)
\item[SNN] 随机神经网络(Stochastic Neural Network)
\end{denotation}



%%% 正文部分
\mainmatter
%\include{data/chap01}
%\include{data/chap02}
\chapter{介绍}
近年来,在人工智能领域,强化学习得到了极为广泛的重视和研究。强化学习是一种常见的人工智能范式,旨在解决智能体和环境交互中的决策问题。在这篇论文中,我们为解决一类特殊的问题(``稀疏奖励问题'')而提出了一种分层强化学习的架构。我们提出了一个新的分层强化学习算法``基于优势函数辅助奖励的分层强化学习''(Hierarchical reinforcement learning with Advantage-based Auxilliary Reward, HARR),在强化学习社区常用的基准实验环境(Benchmark\cite{benchmarking_RL})中进行了实验测试,取得了当前学术界的最佳成绩(在公平对比的条件下)。在此基础上,我们把这套算法应用于自动驾驶仿真环境,解决了一个自动驾驶路径和速度规划的问题。

接下来,我们将会首先介绍强化学习的一般问题,其中将重点介绍分层强化学习这一重要分支,并由此引出强化学习中有待研究的问题,指出我们具体研究的问题及其意义。

\section{强化学习预备知识}
  在强化学习的著作\textit{Reinforcement Learning An Introduction}\cite{Sutton_book}中,Sutton对强化学习的理论进行了详尽的讨论。在这里,我们对强化学习的发展做一个精炼的综述,并从契合课题的框架进行一些讨论。另外,我们还将详细讨论强化学习的一些最新进展和它们在本课题中的应用。
  \subsection{问题描述}
    Sutton在1988年首次提出强化学习问题的一般描述\cite{Sutton_problem_formulation}。诸如学习走路、学习下棋的一般性问题都可以归为强化学习的问题。强化学习问题的四个基本元素是:一个策略(policy),一个奖励函数(reward),一个价值函数(value function)和一个环境模型(model)。例如,自动驾驶问题中,智能体面对它观察到的一个周边车流的\textbf{环境},会给出一个输出,例如以某一加速度加速。在采取这一行动之后,智能体会得到一个\textbf{奖励}(可能有延迟),例如它到达了目的地,我们给它价值$100$的奖励。而如何决定加速度,就是我们想要智能体去学习的\textbf{策略}。在制定策略的过程中,智能体会去试图衡量目前所处状态的\textbf{价值}(价值函数),例如周围车辆非常拥堵,这个状态的价值就比较低,是我们希望避免进入的状态;而车前非常空旷则可能是我们赋予价值较高,希望进入的状态。\par
    在这里,我们将用一些研究者所默认的记号对我们的问题进行数学抽象。假设下标$t$表示任一时刻,而$S_t$代表某一时刻的状态,$A_t$表示智能体采取的行动,$\pi$表示智能体的策略,$R_t$表示智能体某一步获得的奖励。我们的目标是让智能体通过学习,获得尽可能大的总奖励,记为$G$。有时我们可以定义
      $$ G = \sum_{t} R_t $$
    但是多数时候我们会考虑加入一个折扣$\gamma$($0 < \gamma < 1$),认为一个行为直接获得的奖励对它更有意义,而很久以后获得的奖励与这个行为的关联越来越小,记为
      \begin{align}
        G_{t_0} = \sum_{t = t0}^{t_n} \gamma^{t-t_0} R_t
        \label{eq:G}
      \end{align}
    \par 其中$t_n$要么是一个过程因为某种原因终止的时刻,要么是$\infty$
    \par
    强化学习解决的是一个马尔可夫决策过程(MDP)\cite{Bellman_MDP}。在MDP中,$S_{t+1}$只取决于$S_t$和$A_t$,而不受到之前历史过程的影响。也就是说,$S_t$状态本身就包含了它所有的历史对后续过程的影响。在这个假设下,我们可以这样表达强化学习的目标\cite{DRL_for_driving}:记$v^{\pi}(s)$为状态$s$的价值函数。我们有$$ v^{\pi}(s) = E[G_{t_0} | S_{t_0} = s, \pi(s) ] $$
      $$ J(\pi) = \max_{\pi} E(v^{\pi}(s)) $$
      $$ \pi^*(s) = \argmax_{\pi} J(\pi) $$
    \par 强化学习的终极目标就是找到这个$\pi^*(s)$。Bellman在\textit{Dynamic Programming}一书中提出了著名的Bellman最优性等式\cite{Bellman_DP}
    \begin{align}
       v_*(s) = \max_{a} \sum_{s', r}p(s', r | s, a)[r + \gamma v_*(s')]
    \end{align}
    \par 同时,我们用$q(s, a)$表示在状态$s$下采取行动$a$可以获得的价值,那么Bellman最优性等式又可以表示为
    \begin{align}
       q_*(s, a) = \sum_{s', r}p(s', r | s, a)[r + \gamma \max_{a'} q_*(s', a')]
    \end{align}

    \par 这两个表达式在后续的推导中会非常有用。
    
  \subsection{强化学习典型算法综述}
      强化学习包含许多的算法,也有很多分类方法。其中,有一大类是列表方法,可以理解为是把$v(s)$保存起来的方法;另一大类是近似方法,也就是用一个函数去近似$v(s)$\cite{Sutton_book}。其中,第二类方法在近年来使用最多,神经网络使得函数可以近似非常复杂的映射关系,也让许多复杂问题通过强化学习得到了解决。\par
      列表式的强化学习方法包括动态规划\cite{Bellman_DP},蒙特卡罗方法,时序差分学习(Temporal-difference learning,简称TD方法,包括Sarsa\cite{Sutton_book}, Q-learning\cite{Q_learning})\cite{Sutton_problem_formulation},多步自助法\cite{n_step_bootstrapping}(n-step bootstrapping),Dyna规划方法\cite{Sutton_book}等。\par
      TD方法是列表法强化学习的核心。这一族的算法有许多衍生版本,但其核心是利用经验不断更新价值函数$V(s)$,直至其收敛。更新的基本模式为
      $$ V(s) \gets V(s) + \alpha [r + V(s') - V(s)] $$
      \par 其中,$\alpha$表示函数更新的步幅。而当我们用$Q(s, a)$去替代$V(s)$的时候,算法就成为了Sarsa(state, action, reward, state, action \cite{deepRL_overview}):
      $$ Q(s, a) \gets Q(s, a) + \alpha [r + \gamma Q(s', a') - Q(s, a)] $$
      \par 注意这里,$a'$是依据当前策略$\pi$采取的行动。而当我们把$Q(s', a')$替换成$\max_{a'} Q(s', a')$的时候,算法就成为了Q-学习法(Q-learning):
      \begin{align}
      \label{eq:Q_learning}
        Q(s, a) \gets Q(s, a) + \alpha [r + \gamma \max_{a'} Q(s', a') - Q(s, a)]
      \end{align}
      \par 在这里,我们发现Sarsa和Q-学习的重要区别:Sarsa是同策略学习的(on-policy),估算$Q$值时采用的策略就是当前策略,而Q-学习是采用异策略(off-policy)学习的,估算$Q$值时采用的策略不是当前策略,而是最优策略。同策略与异策略也作为强化学习中的两种并行方法,本身并不改变算法架构,但是会影响收敛性等特性。许多算法都可以在这个维度上进行变异。\par
      注意到在以上的算法中,我们每一步都对$Q$或者$V$进行更新。在多步骤自助法(n-step bootstrapping)中,我们则将更新进行延迟。例如,某一行动结束,$n$步过后,我们才观察这一行为造成的影响,并且对它的值进行更新,其更新方式如下
      $$ V(S_t) \gets V(S_t) + \alpha(G_{t:t+n} - V(S_t)) $$
      \par 其中,$G_{t:t+n} = \sum_{i = 1}^{n}\gamma^{i-1}R_{t+i} + \gamma^n V(S_{t+n})$
      \par Sutton等人指出,多步骤自助法强化学习在训练时的表现往往比蒙特卡罗和TD方法都要更优\cite{Sutton_book}。
      \par 归功于深度强化学习的提出\cite{atari_2013}\cite{deepRL_overview},近似函数的方法在近年来得到了极大的发展。近似函数有两种基本思路。第一种,是用某一类型的函数去近似状态的价值函数$v(s)$,从而帮助智能体根据不同的$v$采取不同的策略。第二种思路是跳过$v$的计算,而直接使用函数去近似一个策略$\pi$。在第二种思路中,$\pi(s)$的输出将直接是智能体应该采取的行为(或者行为的一个概率分布)。这种方法被称为``策略优化'',围绕策略优化问题衍生出一系列的策略梯度求解方法,成为近年来强化学习的热点研究方向之一。进一步的,在策略优化问题中,如果我们同时对价值函数$v(s)$进行近似和更新,那么可以得到一类框架,我们称之为actor-critic架构\cite{Sutton_book}。在后续的篇目中,我们会对此进行具体介绍。\par
      在深度强化学习中,比较著名的算法有DDPG\cite{DDPG},A2C\cite{A3C},TRPO\cite{TRPO},ACKTR\cite{ACKTR},ACER\cite{ACER},PPO\cite{PPO}等。这些算法全部忽略了环境的模型,或者环境模型是学习得到的(TRPO)。还有一类算法,其中的环境模型是提前设计建模得到的而非习得,其中有代表性的就是AlphaZero\cite{AlphaZero}。这类算法也是我们想探究的方向——是否通过提前建模,能够让自动驾驶车辆在环岛这类复杂环境中有更好的表现?在接下来的章节中,我们会详细介绍其中有代表性,并且在我们的场景下易于使用的算法。\par
      另外,在对价值函数进行优化时(梯度下降),不同的算法可能带来不同的结果。A3C采用了RMSProp算法\cite{RMSProp},我们往往还可以选择随机梯度下降法(SGD),带动量的随机梯度下降法\cite{SGD_momentum}等方法,在算法设计时需要进行尝试和选择。Sutton在书中对此进行了讨论\cite{Sutton_book}。此外,Ruder在2016的一篇综述文中对各种梯度下降算法进行了详尽的探讨\cite{GD_overview}。
      
    \subsection{无模型的深度强化学习}
      无模型的深度强化学习是我们在本课题中的重要算法基础之一。其中,2017年被提出的TRPO算法\cite{TRPO}是我们在本课题中的重点。基于TRPO,我们构建了分层结构,并且基于TRPO中策略单调提升的结论,我们证明了HAR算法的单调性。因此接下来我们将介绍无模型的深度强化学习问题及其典型算法。
      
      Mnih等人在2013年首度提出深度Q-网络(deep Q-network,简称DQN)\cite{atari_2013},2015年,他们的另一篇论文同样描述了DQN\cite{nature2015},激发了一轮深度神经网络与强化学习结合的热潮\cite{deepRL_overview}。DQN的基本思路是创造一个很一般的端到端的网络$Q(s, a | \theta)$,其中$\theta$是这个网络的参数。例如对于一个游戏,网络的输入直接就是未处理的图像,输出直接就是在某个阶段,游戏中的动作对应的价值。在传统Q-learning的框架下,DQN在每一步更新$q(s, a)$值的时候,不是直接使用\eqref{eq:Q_learning},而是利用梯度下降法去更新网络$Q$的参数$\theta$,使得$Q$估值总损失最小
      \begin{align*}
        \bigtriangledown_{\theta_i}Loss(\theta_i) = E_{s, a, s'}[(r+\gamma \max_{a'}Q(s', a'|\theta_{i-1}) - Q(s, a|\theta_i))\bigtriangledown_{\theta_i}Q(s, a|\theta_i)]
      \end{align*}
      这里DQN的网络结构参考了Hinton等人提出的AlexNet\cite{AlexNet},也就是典型的卷积神经网络(CNN)结构。DQN的最大贡献在于,由于使用了经验回放的方法,DQN可以收敛,从而使得深度神经网络在强化学习中的应用成为了可能。由于神经网络的高度非线性,DQN得以模拟足够复杂的场景。DQN在Atari系列游戏中取得了突破性的进展,甚至在一些游戏中超越了人类专家的水平。\par
      不过由于我们想探究的问题更多地涉及到低维度的传感器信息输入,而不是高维度的图像直接输入,因此DQN一类的方法对我们的帮助并不是很大。而策略梯度方法则对我们有很大的借鉴意义,因此我们也将集中讨论这一类的方法。\par
      前面我们提到,在策略优化问题中,有一类架构被称为actor-critic架构。Mnih等人在2016年提出的A3C(asynchronous advantage actor-critic)算法\cite{A3C}成为应用actor-critic架构的经典案例。A3C算法同时维护一个策略函数$\pi(a|s; \theta)$和一个价值函数的估计$V(s; \theta_v )$,这两个函数均为卷积神经网络CNN,并且两个网络除了输出层以外的层均是共享的。其中, $\pi(a|s; \theta)$称为actor,而$V(s; \theta_v )$称为critic。actor-critic会计算一个``优势'',定义为$\delta = R + \gamma V(S'; \theta_v) - V(S; \theta_v)$\cite{Sutton_book}。A3C借鉴了多步骤自助法的思路(设步数为$k$),在计算优势时,变异为$\delta_t = \sum_{i = 0}^{k - 1}\gamma^i r_{t+i} + \gamma^k V(S_{t+k}; \theta_v) - V(S; \theta_v)$。每一步,我们依然对$\theta$和$\theta_v$进行更新,梯度的计算方法如下(注意计算梯度以后,还需要选择一个梯度下降方法对$\theta$和$\theta_v$进行更新)
      \begin{align}
      \label{eq:A3C_gradient1}
          d\theta &= \delta_t \bigtriangledown_{\theta} log_\pi(A_t|S_t; \theta)\\
      \label{eq:A3C_gradient2}
          d\theta_v &= \delta_t \bigtriangledown_{\theta_v} \delta_t^2
      \end{align}
      \par 如果我们采用多个智能体并行训练,把它们各自的梯度叠加在一起(也就是累加每个智能体各自求得的\eqref{eq:A3C_gradient1}\eqref{eq:A3C_gradient2}式),最后统一更新$\theta$和$\theta_v$,就称之为异步训练方法,这种算法也因此被称为asynchronous advantage actor-critic。如果条件受限没有异步,而只是用一个智能体进行训练,这个算法就称为A2C(advantage actor-critic)。actor-critic架构被认为是能够有效加速训练的,因此越来越多的算法都基于这一架构进行设计。\par
      2015年,Schulamn等人提出了经典的(Trust Region Policy Optimization, TRPO)算法\cite{TRPO}。在文章中,他们首先从理论上证明了一种策略迭代更新的MM(minorization-maximization)算法,可以确保策略对应的总价值($\eta(\pi) = E_{s_0}[\sum \gamma^t r(s_t); \pi_\theta]$)单调不下降。依据这一理论基础,他们进行了多步近似,将MM算法中的优化(求$\argmax$)的问题转化为了一个更简单的加限制的优化问题(信任空间优化,trust-region optimization)。在进行进一步近似后,优化式中的期望形式可以用蒙特卡罗方法进行估算。最后,优化式本身则利用一种近似的方法进行优化。在实验中,他们采用了神经网络作为策略$\pi$的表达。TRPO的更新方法是:
      \begin{align}
        \begin{aligned}
          \theta = \argmax_\theta E\bigg[\frac{\pi_\theta(a|s)}{\pi_{\theta_{old}}(a|s)}Q_{\theta_{old}}(s, a)\bigg]\\
          \text{subject to } E[D_{KL}(\pi_{\theta_{old}}(\cdot|s)||\pi_\theta(\cdot|s))] \leq \delta
        \end{aligned}
      \end{align}
      \par 其中,$\delta$是一个认为选定的小值。注意,在论文的附录中,作者提出采用一阶近似对上式进行求解,优化问题就可以就变为一个典型的拉格朗日共轭问题\cite{Convex_optimization_book}。不过,由于TRPO只是保证了每一步更新,策略都变得更好,却没有考虑探索性,因此,TRPO的结果很容易困在局部最优值。在Schulman之后,Wu等人于2017年又提出了一个改进算法ACKTR\cite{ACKTR}(Actor Critic using Kronecker-Factored Trust Region),采用actor-critic架构,并利用Kronecker因子去近似梯度下降。同年,Schulman等人又提出了PPO算法\cite{PPO},在实现难度和算法表现方面取得了平衡。PPO算法的基本思路有三点:
      \begin{itemize}
        \item 去除TRPO的优化限制条件,在TRPO的基础上,采用一个clip函数,从而阻止$\theta$每一次更新太大,确保收敛
        \item 在优化的函数里,通过引入系数的形式,同时优化actor和critic,整个优化过程只有一个优化目标函数
        \item 通过在优化目标中加入一个熵奖励,增加算法的探索性
      \end{itemize}
      \par PPO把强化学习问题转化为一个形式简单的单一函数优化问题,在许多方面取得了比以往算法(A2C, ACER, TRPO, etc.)更优的表现。\par
      在2015年,Silver首度提出了针对连续动作空间的确定性算法DPG(deterministic policy gradient)\cite{DPG}。与确定性算法对应的是随机算法,也就是策略是行动的概率分布。在高维的动作空间内(例如动作是连续的),DPG比与之对应的随机算法获得了更优的结果。DPG采用actor-critic架构,并且是异策略(off-policy)训练的。注意这里,DPG的actor-critic架构与Sutton书中提到的actor-critic架构有所区别。它是不以价值函数作为基准的,而正常的actor-critic架构应该以价值函数作为基准,计算优势,再进行求导\cite{Sutton_book},参见前面提到的A3C算法\cite{A3C}。同时,DPG算法采用的是Q-learning结构,不是用$V(s)$对价值进行估计,而是用$Q(s, \pi(s))$对价值进行估计。由于算法本身是确定性的,DPG与随机异策略actor-critic架构\cite{off-policy_actor-critic}不同,不需要针对策略进行重要性采样(importance sampling,在\cite{Sutton_book}中有详细讨论)。DPG本身没有选择神经网络,而是可以采用一般的函数,尤其是线性函数。\par
      DPG算法对$\pi(s)$和$\mu(s)$的更新方法如下:
      \begin{align}
        \begin{aligned}
          &\delta_t = r_t + \gamma Q(S_{t+1}, A_{t+1};\theta_Q) - Q(S_t, A_t;\theta_Q)\\
          &\theta_Q \gets \theta_Q + \alpha_{\theta_Q}\delta_t \bigtriangledown_{\theta_Q}Q(s, a;\theta_Q)\\
          &\theta_\mu \gets \theta_\mu + \alpha_{\theta_\mu}\bigtriangledown_{\theta_\mu}\mu(s;\theta_\mu)\bigtriangledown_{a}Q(s, a;\theta_Q)|_{a = \mu(s;\theta_\mu)}
        \end{aligned}
        \label{eq:DPG_update}
      \end{align}
      \par 基于DPG,Lillicrap等人提出的改进算法,DDPG算法\cite{DDPG},则全面采用了深度神经网络。为了能够让训练收敛,DDPG使用了一个目标网络(target network)的概念。目标网络的定义是缓慢地跟踪依据梯度下降优化得到的网络参数$\theta_Q$和$\theta_\mu$,$\theta_Q$是状态行动对的价值估计网络$Q$的参数,而$\theta_\mu$是actor-critic架构中actor采用的策略的参数。目标网络参数$\theta_Q'$和$\theta_\mu'$的更新方法如下(在每一步迭代过后进行这个更新):
      \begin{align*}
        \theta' = \tau \theta + (1-\tau) \theta', \tau \ll 1
        \label{eq:target_network}
      \end{align*}
      \par 这样,相当于强迫目标网络缓慢地变化,从而增强了训练的收敛性。目标网络主要用于带入\eqref{eq:DPG_update}中$\delta_t$的计算。同时,DDPG采用了经验回放的方法,在每一步迭代中都采用之前历史中一批随机挑选的过往经验进行批量计算,而非一步的结果,这样可以确保网络训练时不带有时序偏差。另外,DDPG没有像DPG一样采用异策略学习,而是在策略基础上加上一个噪声分布以确保``探索''性。\par
      
    \subsection{基于模型的强化学习}
    基于模型的强化学习并非本课题的重点,但是是强化学习问题当中重要的一部分,和无模型的强化学习算法并驾齐驱。目前,分层强化学习问题多数基于无模型的强化学习算法,很少考虑模型问题。但是实际上,模型可以大大提高算法的成功率、采样效率和稳定性。为完整性考量,在这里我们也对基于模型的强化学习算法进行简要的介绍,也作为给未来研究方向的一个灵感。
    
    基于模型的深度强化学习问题可以分为模型是``习得''的和模型是给定的两种。在这里,我们首先关注一下给定模型的强化学习的应用。
      \par 在模型给定的算法中,最著名的例子就是2016年提出的AlphaGo\cite{AlphaGo}。Deepmind在2017年进一步提出了AlphaZero\cite{AlphaZero},其算法和AlphaGo相比有几大特点:
      \begin{enumerate}
        \item AlphaZero完全采用逐步迭代的方式,而AlphaGo在训练中会挑选之前出现过的最好策略
        \item AlphaZero拥有的先验知识只有棋盘的格局和下棋的规则,并且这些知识只用来进行模拟
        \item AlphaZero对参数的变动更加鲁棒,不同棋类游戏用的都是同一套参数
        \item AlphaZero在训练中没有对输入进行任何数据加强的操作,而AlphaGo则对棋盘进行了镜像操作
      \end{enumerate}
      \par AlphaZero采用蒙特卡罗树状搜索(Monte Carlo Tree Search, MCTS\cite{Sutton_book})的方法,每次优先访问训练历史中访问次数低,但访问概率$\pi(a|s)$应该较高,且$Q(s, a)$的状态行为对$(s, a)$。同时,在此基础上加上一个噪声(在AlphaZero的例子中,一个狄利赫里分布的噪声),确保探索性。AlphaZero采用深度卷积神经网络作为策略$\pi(s)$的拟合函数,其输出为$\pi(s) = (\mathbf{p}, v)$,其中,$\mathbf{p}$为在某个状态下选择各个行为的概率,$v$为根据$s$推测的状态估值,$v = \mathbb{E}[z]$(胜负平局,$z$的值分别为$+1, -1, 0$)。针对该网络定义的损失函数为
      \begin{align}
        L = (z-v)^2 - \pi^T log \mathbf(p) + \lambda ||\theta||^2
      \end{align}
      \par 可以看出,这个损失函数试图使$v$的预测和棋局结果$z$更加接近;$\pi$代表了训练中各个动作的搜索概率,损失函数试图让神经网络的输出$\mathbf(p)$与之尽可能接近。
      \par 这里所谓“给定模型”,其实就是环境是已知的,不需要真的和环境交互,而是在仿真里直接进行下棋这个活动即可。然而,这也意味着这个问题本身与直接和未知环境交互并没有区别。尽管模型给定,算法并没有从模型本身得出什么规律,因此这个模型本质上就是一个环境。\par
      在强化学习问题中,从与环境的交互历史中逐渐学习到模型,则是另一个有趣的思路。在\cite{Sutton_book}中,Dyna架构就是这样一种思路。在强化学习中,利用模型去近似求解$v(s)$是一个``规划''(planning)问题。传统的列表式方法可以根据训练中得到的环境反馈不断更新模型。2015年,Oh等人在\cite{action-conditional_prediction}一文中提出了一种利用神经网络学习高维度环境的方法,使得复杂环境的模拟也成为了可能。\par 
      基于\cite{action-conditional_prediction}的结果,Racani\`{e}re等人在论文\cite{I2A}中提出了一种利用模型去增强无模型强化学习的方案,简称为I2C算法。这种方法本质上是细化了的A3C算法。注意到A3C本身是一个无模型的方法,它仅仅依赖于$\pi(a|s)$和$v(s)$的估计(回顾一下\eqref{eq:A3C_gradient1}\eqref{eq:A3C_gradient2}两个更新式)。但是,I2C算法修改了$\theta$的组织形式,构建了一个较为复杂的网络结构,因而引入了环境模型。其基本架构如图\ref{fig:I2A}所示。
      \begin{figure}[H] % use float package if you want it here
        \centering
        \includegraphics{I2A}
        \caption{I2A的基本结构:a) 一个预先训练好的神经网络,可以预测在策略$\hat{\pi}$下,状态$s$的一系列时序变化;b) 一个子网络,用于``理解''a)的输出结果;c) 整体架构}
        \label{fig:I2A}
      \end{figure}
      \par 图\ref{fig:I2A}c)展示了网络整体架构,可以看到整体网络的输入为$o_t$,也就是对状态$S_t$的观测,输出为策略$\pi$和价值函数估值$V$(用于actor-critic架构)。因此,这个结构满足A3C的基本框架。在框架内部,网络结构采用了含有预测模型网络处理的输入,也采用了无模型网络处理的输入,分别作为两个分支,最后通过一个子网络汇合。具体的结构参数可以参见论文\cite{I2A}的附录。通过引入这个特殊的网络结构,I2A算法将习得的模型结合进了原有的无模型学习框架中。同时,实验结果表明,这里的模型可以有错误和不精确的地方,学习结果依然是鲁棒的。
    
\section{分层强化学习问题构建}
分层强化学习算法解决的是这样一类特殊的强化学习问题:

在本篇论文中,分层强化学习要解决两个主要问题:
\begin{enumerate}
  \item 如何利用分层结构,让环境的稀疏奖励尽可能密集地扩散给处于早期探索阶段的智能体。
  \item 如何利用上层策略,合理地整合多种不同的底层策略,成为一个新的复合策略。
\end{enumerate}

\section{分层强化学习研究现状}
\subsection{常见算法思路对比}
分层强化学习的研究分为几个不同的思路。其中一个思路是,训练一个上下层的策略结构,上层策略需要给下层策略提供一个子目标(subgoal),下层策略朝向这个子目标去走,并且依据与自身状态和子目标相关的某个函数,而获得一个内部反馈(intrinsic reward),而上层策略则获得环境给出的真实反馈(extrinsic reward)。这个产生内部反馈的函数,有不同的选择。例如在高效分层强化学习(HIRO\cite{HIRO})算法中,这个函数是下层状态与智能体目标状态的欧氏距离。在HAC\cite{HAC}算法中,这个???。而在层级制网络强化学习算法(FeUdal Network\cite{feudal})当中,这个函数是下层行动方向与上层给出目标方向的余弦距离。同时在FeUdal Networks这篇论文当中,下层策略也获得了外界环境的奖励,这与内部反馈一起加权处理后成为底层策略获得的总的奖励,亦即$R_{low} = R_{extrinsic} + \alpha R_{intrinsic}$。注意到在分层强化学习的这种思路当中,上下层策略都是一起从零开始训练的。

另一种思路是对上下层策略分开训练。在基于随机神经网络的分层强化学习算法(SNN for HRL\cite{SNN4hrl})论文中,作者们提出了一种典型的分层架构:利用随机神经网络构建上层策略,输出1-hot向量(一个一维向量$[x_1, x_2, ..., x_n]^T$,其中只有一个元素为$1$,其余均为$0$),来表示选择哪一个下层策略。下层策略的输入为环境的观察值$o_t$和1-hot向量$v$,输出值是智能体动作空间的行动。为了获得可用的下层策略,SNN for HRL要求进行底层策略预训练。预训练法是分层强化学习当中常用的一种算法。它的基本思路是,在一个比较简单的、环境反馈(即奖励值)比较密集的环境中训练下层策略。由于实际策略在操作的时候不需要获得``奖励'',因此只要观测空间和动作空间的维度相同,下层策略就可以直接在不同环境中迁移。SNN的一大贡献在于,为了能够训练出需要1-hot向量输入的下层策略,引入了一个共享信息熵的奖励(Mutual-Information, MI bonus)。

SNN for HRL的问题在于,下层策略一旦训练出来,就不可以再改变。文章中作者指出,有一种潜在的方法可以联合训练上下层策略,而这种方法是基于对整体网络的梯度下降\cite{categorical_gradient}。然而,正如FeUdal Network\cite{feudal}一文指出的,这样一类直接对整体网络结构进行上下层一起优化的算法可能会丢失上层输出的语义信息,造成训练效果不佳。因此,我们的方法决定采用上下层同时训练,但分成两次优化。这样,我们可以确保上下层策略具有足够的解离度,是可解释的。???看一下论文再修改这里。

\subsection{马尔可夫过程的可观测性}
尽管上述许多算法能够很好地完成基准实验环境下强化学习的任务,但是这些算法在机器人的实际操作中存在很大的问题。其中,马尔可夫过程中环境的观测性是一个很大的问题。

上述HAC\cite{HAC},HIRO\cite{HIRO}等算法在设计下层奖励值的时候,都用到了欧氏距离。欧氏距离有一个很大的问题在于我们需要获取环境的真实表现(ground truth)而非仅仅观察值。例如,一个机器人在迷宫中行走,如果想要采取上述方法,那么就需要机器人知道自身的$(x, y)$坐标。实际上,这个坐标时很难获取的。也就是说,实际当中,智能体所处的马尔可夫过程并非完全可观测。考虑到如果智能体完全看不到目标,就永远无法知道应该往哪里走,因此在实际操作中,我们减小难度,让智能体能够穿透墙面看到目标。理论上,对于一个不是非常复杂的迷宫,智能体和目标的相对位置与智能体的观察值是一一对应的。但实际当中,这个对应关系非常难以恢复。换言之,我们可以说智能体面临的仍然是一个较难观测的马尔可夫问题。

事实上,在2019年的一篇文章\cite{sensitive_to_goal_space}中,作者通过实验指出,这些基于上层给下层目标的分层强化学习都面临着``目标空间''的定义难题。文中作者对这些论文给出的实验环境修改了状态的表示(也就对应着``目标空间''的定义),经过一些细微的变异,这些算法就无法学出来。经过试验,我们也论证了HAC\cite{HAC},HIRO\cite{HIRO}都不能在这样的观测下成功。这一点是可以解释的。当观测目标由欧式空间的$(x, y)$坐标变为其他表示时,欧氏距离的远近不再能够表征智能体实际距离目标的远近,因此这些算法在非欧式观察空间下是不能成功。而恰巧机器人应用场景多数都无法直接观察自身的位置(除非通过SLAM一类的位置估计算法进行地图构建,不在我们的讨论范围内),因此这些算法在实际中往往是不实用的。

\subsection{与HAAR相关的工作}
在本部分,我们将介绍一些启发了我们的HAAR算法的,或是与我们研究的问题相关的工作。

\section{论文结构}

\documentclass[degree=bachelor, tocarialchapter, pifootnote]{thuthesis}
% 选项
%   degree=[bachelor|master|doctor|postdoctor], % 必选,学位类型
%   secret,                % 可选(默认:关闭),是否有密级
%   tocarialchapter,       % 可选(默认:关闭),章目录中使用黑体(这项表示同时打开下面两项)
%   tocarialchapterentry,  % 可选(默认:关闭),单独控制章标题在目录中使用黑体
%   tocarialchapterpage,   % 可选(默认:关闭),单独控制章页码在目录中使用黑体
%   pifootnote,            % 可选(默认:关闭),页脚编号采用 pifont 字体符号,建议打开

% 所有其它可能用到的包都统一放到这里了,可以根据自己的实际添加或者删除。
\usepackage{thuthesis}

% 下面的包让argmax之类的,_{}的东西显示在argmax正下方
\usepackage{amsmath}
\DeclareMathOperator*{\argmax}{argmax}
\DeclareMathOperator*{\argmin}{argmin}

% 定义所有的图片文件在 figures 子目录下
\graphicspath{{figures/}}

% 可以在这里修改配置文件中的定义。导言区可以使用中文。
\def\myname{王芮}

\begin{document}

%%% 封面部分å
%\frontmatter

%\thusetup{
  %******************************
  % 注意:
  %   1. 配置里面不要出现空行
  %   2. 不需要的配置信息可以删除
  %******************************
  %
  %=====
  % 秘级
  %=====
%  secretlevel={秘密},
%  secretyear={10},
  %
  %=========
  % 中文信息
  %=========
  %ctitle={分层强化学习算法及其在机器人仿真环境中的应用研究},
  ctitle={HAAR:基于优势函数辅助奖励的分层强化学习算法},
  cdegree={工学学士},
  cdepartment={机械工程系},
  cmajor={机械工程},
  cauthor={王芮},
  csupervisor={雷丽萍副教授},
%  ccosupervisor={某某某教授}, % 联合指导老师
  % 日期自动使用当前时间,若需指定按如下方式修改:
  % cdate={超新星纪元},
  %
  % 博士后专有部分
%  cfirstdiscipline={计算机科学与技术},
%  cseconddiscipline={系统结构},
%  postdoctordate={2009年7月——2011年7月},
%  id={编号}, % 可以留空: id={},
%  udc={UDC}, % 可以留空
%  catalognumber={分类号}, % 可以留空
  %
  %=========
  % 英文信息
  %=========
%  etitle={An Introduction to \LaTeX{} Thesis Template of Tsinghua University v\version},
  % 这块比较复杂,需要分情况讨论:
  % 1. 学术型硕士
  %    edegree:必须为Master of Arts或Master of Science(注意大小写)
  %             “哲学、文学、历史学、法学、教育学、艺术学门类,公共管理学科
  %              填写Master of Arts,其它填写Master of Science”
  %    emajor:“获得一级学科授权的学科填写一级学科名称,其它填写二级学科名称”
  % 2. 专业型硕士
  %    edegree:“填写专业学位英文名称全称”
  %    emajor:“工程硕士填写工程领域,其它专业学位不填写此项”
  % 3. 学术型博士
  %    edegree:Doctor of Philosophy(注意大小写)
  %    emajor:“获得一级学科授权的学科填写一级学科名称,其它填写二级学科名称”
  % 4. 专业型博士
  %    edegree:“填写专业学位英文名称全称”
  %    emajor:不填写此项
%  edegree={Doctor of Engineering},
%  emajor={Computer Science and Technology},
%  eauthor={Xue Ruini},
%  esupervisor={Professor Zheng Weimin},
%  eassosupervisor={Chen Wenguang},
  % 日期自动生成,若需指定按如下方式修改:
  % edate={December, 2005}
  %
  % 关键词用“英文逗号”分割
  ckeywords={强化学习, 分层强化学习, 机器人学, 规划算法, 自动驾驶, 路径规划, 优化, 策略单调性},
  ekeywords={Reinforcement Learning, Hierarchical Reinforcement Learning, Robotics, Planning Algorithms, Self-Driving, Path Planning, Optimization, Monotonic Improvement}
}

% 定义中英文摘要和关键字
\begin{cabstract}
强化学习问题中的一个重要挑战是环境的稀疏奖励问题。如果环境反馈给智能体的信号过于稀疏,智能体的学习速度会非常缓慢。为此,人们提出分层强化学习的方法,来帮助强化学习算法提高采样效率,加速学习。在分层强化学习范式当中,特别地,我们需要为下层策略设计一个辅助奖励函数,从而促进下层策略的学习。在本篇论文中,我们提出了一个分层强化学习的框架,通过基于优势函数的下层奖励函数设计,它可以单调地同时优化上下层策略。截止目前,绝大多数最前沿的分层强化学习算法都对状态空间的表达非常敏感,或者依赖于手动设计的、只在特定环境下可用的下层奖励函数。得益于我们的辅助奖励函数设计只与优势函数的值相关这一特点,我们的算法不依赖于状态空间的表示方式,因此和其他方法相比更为通用和实用,尤其是在机器人连续动作空间这类应用场景。我们从理论上证明了这种算法可以保留上下层各自优化算法的单调特性,使整体策略也得到单调优化。实验方面,在Mujoco仿真环境的多个任务中,我们的算法表现均大大优于现有的最佳算法。最后,我们还搭建了一个自动驾驶仿真环境来验证本算法在更现实和复杂的环境下的表现。我们设计了迁移实验,将算法从Mujoco实验环境迁移到了自动驾驶环岛路径规划问题中,同时也指出了在这个环境中的下一步研究方向。
\end{cabstract}

\begin{eabstract}
One of the key challenges in reinforcement learning is related to sparse reward signals from the environment. Hierarchical reinforcement learning (HRL) can be utilized to solve the credit assignment problem in a sparse reward task. In the HRL paradigms, we need to design a low-level auxiliary reward function to facilitate low-level learning. In this paper, we propose an HRL framework which monotonically  optimizes the joint policy of the upper-level and lower-level  through the design of an advantage function-based auxiliary reward. Most state-of-the-art HRL algorithms are sensitive to state representations, or require careful design of the auxiliary reward functions  for low-level training. Our method is more generic and applicable in continuous control tasks since our auxiliary reward for low-level policy update is dependent only on the advantage function of the high-level policy, therefore invariant of the state representations. We theoretically prove that monotonic improvement of the joint-policy is guaranteed in our method. Experimental results show that our algorithm performs dramatically better than other state-of-the-art methods across multiple tasks in the Mujoco learning environment. Finally, we design a complex autonomous-driving simulation environment as a testbed for our algorithm in more realistic settings. We transfer our algorithm to this new environment and report the experiment results of our framework on the self-driving car's planning task at a roundabout, as well as point out the direction for future research.

\end{eabstract}

% 如果习惯关键字跟在摘要文字后面,可以用直接命令来设置,如下:
% \ckeywords{\TeX, \LaTeX, CJK, 模板, 论文}

% \ekeywords{\TeX, \LaTeX, CJK, template, thesis}

% 如果使用授权说明扫描页,将可选参数中指定为扫描得到的 PDF 文件名,例如:
% \makecover[scan-auth.pdf]
%\makecover
%% 目录
% 设置目录层级,三级小标题全部显示
%\setcounter{secnumdepth}{5}
%\setcounter{tocdepth}{5}
%\tableofcontents

%% 符号对照表
%\begin{denotation}[3cm]
\item[$s$] 状态(任意状态)
\item[$a$] 行动(任意行动)
\item[$\pi$] 策略
\item[$\pi(s)$] 策略函数
\item[$\pi(a|s)$] 策略在状态$s$下采取行动$a$的概率
\item[$r$] 奖励
\item[$r(s, a)$] 在状态$s$下采取行动$a$获得的奖励
\item[$G_t$] 时间$t$之后的总奖励
\item[$v(s)$] 状态$s$的价值函数
\item[$q_\pi(s, a)$] 在策略$\pi$下,在状态$s$下采取行动$a$的价值
\item[$V(s)$] 状态$s$的价值函数的估计值
\item[$Q(s, a)$] 状态$s$,动作$a$的价值的估计值
\item[$p(s'|s, a)$] 在状态$s$采取动作$a$,状态转移到$s'$的概率
\item[$p(s', r|s, a)$] 在状态$s$采取动作$a$,状态转移到$s'$并且获得奖励$r$的概率
\item[$\pi(a|s, \theta)$] 策略函数的参数是$\theta$,在状态$s$下采取行动$a$的概率
\item[$\pi_\theta$] 策略函数,其参数为$\theta$
\item[$\bigtriangledown \pi(a|s, \theta)$] 策略函数的参数是$\theta$,在状态$s$下采取行动$a$的概率对$\theta$的导数
\item[$\gamma$] 总奖励$G_t$计算时,对于后续$R_{t+t_i}$的折扣率
\item[$\rho$] 重要性采样比例(importance-sampling ratio)
\item[$\delta_t$] 时序差异误差(TD error)
\item[RL] 强化学习(reinforcement learning)
\item[MPC] 模型预测控制(model-predictive control)
\item[TD] 时序差异(temporal-difference)
\item[MDP] 马尔可夫过程 (Markov Decision Process)
\item[POMDP] 部分可观测的马尔可夫过程 (Partially Observable Markov Decision Process)
\item[DQN] 深度Q-学习神经网络(Deep Q Network)
\item[DRQN] 深度循环Q-学习神经网络(Deep Recurrent Q Network)
\item[SNN] 随机神经网络(Stochastic Neural Network)
\end{denotation}



%%% 正文部分
\mainmatter
%\include{data/chap01}
%\include{data/chap02}
%\renewcommand\thesection{\arabic {section}} % use this command to disable chapter numbering

\section{Derivation}
我们的算法是分别对上下层策略进行更新,接下来我们证明,单独更新上层策略和单独更新下层策略都可以使得整体策略变优。

在接下来的推导过程中,我们用上标或下标$h$和$l$来表示强化学习中的一系列参数,是针对上层策略(high)还是下层策略(low)。

首先,我们证明,在下层策略不变的情况下,利用TRPO更新上层策略,可以保证整体策略的单调递增,即$V_h(s_0^h)$单调递增。这一步很简单,既然底层策略不变,那么可以认为底层策略是包含在环境中的,因此TRPO去优化上层策略,就相当于TRPO直接作用于一个单层的强化学习结构,对应的优化目标必然是越来越好的。注意到对于上层策略来说,它得到的$R$就是上下层策略作为一个整体得到的真实$R$。

接下来,我们推导,采用$ \frac{\gamma_h V_h(s_{t + k}^h) - V_h(s_{t}^h)}{k}$作为上层给下层的输入,可以使得整体策略变优。

我们的上下层优化算法采用的均是TRPO\cite{TRPO}。在TRPO当中,我们最终希望去优化的东西是$V_{\pi}(s_0)$。TRPO算法等价于在基于策略梯度定理给出的策略梯度优化中,增加了对步长的限制,从而确保策略更新的单调性。在策略梯度方法的论文\cite{policy_gradient_theorem}中,Sutton指出,对一个策略优化的目标函数$J$既可以采用$V_{\pi}(s_0)$的定义,也可以采用$R$的平均值的定义,它们可以分别推导得出策略梯度定理,而这个定理中的策略梯度,与TRPO试图优化的替代函数的梯度是相等的。因此,TRPO也可以理解成是在最大化$\mathbb{E}_{s, a \sim \pi}[R(s, a)]$。

如果把$R$的均值作为优化目标,我们有
\begin{align}
  \bigtriangledown J(\theta) = \bigtriangledown \mathbb{E}_{s\sim\pi}[\sum_{a} \pi(s, a) R(s, a)]
  &= \bigtriangledown \mathbb{E}_{s, a\sim\pi}[R(s, a)].
  \label{eq:policy_gradient}
\end{align}

我们定义了
\begin{align}
  R_l(s_{t + i}^l, a_{t + i}^l)|_{i = 0,1,...,k-1} = \frac{V_h(s_{t + k}^h) - V_h(s_{t}^h)}{k}.
  \label{eq:R_low}
\end{align}

前面我们提到,TRPO完成的任务是最大化$\mathbb{E}_{s, a \sim \pi}[R(s, a)]$,因此我们在利用TRPO针对底层策略进行优化的时候,相当于最大化$\mathbb{E}_{s, a \sim \pi_l, \pi_h}[R_l(s_l, a_l)]$。注意这里,我们把$\pi_h$看做是一个固定的概率分布函数,而不对它进行优化。$\pi_h$可以看做是环境的一部分,而它输出的隐式编码(latent code)则是我们观察(observation)的一部分。

根据\eqref{eq:R_low}中的结果,我们两边取期望,可以得到
\begin{align}
  \mathbb{E}_{s^l, a^l \sim \pi_l, \pi_h}[R_l(s^l, a^l)] = \frac{1}{k} \mathbb{E}_{s^h, a^h \sim \pi_l, \pi_h}[\gamma_h V_h(s_{t + k}^h) - V_h(s_{t}^h)].
  \label{eq:R_expectation}
\end{align}

回忆在强化学习中,我们关于优势函数$A(s, a)$的定义,为
\begin{align}
  A(S_t, A_t) = Q(S_t, A_t) - V(S_t) = R(S_t, A_t) + \gamma V(S_{t + 1}) - V(s).
  \label{eq:advantage}
\end{align}

由于在稀疏强化学习问题中有稀疏奖励条件:
\begin{align}
  R(S_t, A_t) = 0, \forall t \neq t_{end}.
  \label{eq:sparse_reward_condition}
\end{align}

因此,\eqref{eq:advantage}变为
\begin{align}
  A(S_t, A_t) = \gamma V(S_{t + 1}) - V(s).
  \label{eq:sparse_advantage}
\end{align}

由\eqref{eq:R_expectation}和\eqref{eq:sparse_advantage}可知,TRPO优化下层策略的结果,等效于优化了上层的优势函数,也就是
\begin{align}
  \mathbb{E}_{s^l, a^l \sim \pi_l, \pi_h}[R_l(s^l, a^l)] = \frac{1}{k} \mathbb{E}_{s^h, a^h \sim \pi_l, \pi_h}[A^h(s_t^h, a_t^h)].
  \label{eq:R_expectation_is_advantage}
\end{align}

接下来我们证明,通过优化这个期望值,就近似等效于优化了整体策略的表现。这里不能直接把TRPO的(13)搬过来用在上层policy上!TRPO是建立在环境动态不变的基础上的,但是我们的环境变了。必须follow TRPO论文的推导,看一下我们能不能推出一样的结果。下面做的就是这个事情。
%我们已经知道,TRPO最大化的替代函数的梯度,与策略梯度定理指出的梯度成正比。因而这个梯度,对应的需要最大化的函数,可以和TRPO(应用于上层策略)给出的替代函数相同。接下来我们推导在下层策略可能发生改变的情况下,上层策略的目标函数(也就是整体策略的目标函数)如何变化。

注意到,对于上层策略来说,底层策略相当于环境动力学(dynamics)。当底层策略发生了变化时,可以看做是环境动力学发生了变化。因此,即使上层策略不变,$S^h_t$的分布也可能发生改变。我们用$\mathcal{E}(\pi_l)$表示环境,并且环境与$\pi_l$有关。定义$\pi_h$带来的折扣奖励期望
\begin{equation}
\begin{aligned}
  \eta(\pi_h) = \mathbb{E}_{s_0^h, a_0^h, ...}\Bigg[\sum_{t = 0, k, 2k, ...} \gamma_h^{t/k} r_h(s_t^h, a_t^h)\Bigg] \text{ ,  where}   \\
  s_0^h \sim \rho_0^h(s_0^h), a_t^h \sim \pi_h(s_t^h), s_{t+1} \sim P(s_{t+1}^h|s_t^h, a_t^h, \mathcal{E}(\pi_l))
\end{aligned}
\end{equation}

我们用$\tilde{\eta}(\pi_h)$来表示$\pi_l$发生变化(变为$\tilde{\pi_l}$)而$\pi_h$没有发生变化以后的折扣奖励值期望。将TRPO根据\cite{TRPO_pre}得到的引理1(\textbf{Lemma 1})稍加变形,我们可以得到类似的结果:(具体推导参见附录中证明1)
\begin{align}
  \tilde{\eta}(\pi_h) = \eta(\pi_h) + \mathbb{E}_{s_0^h, a_0^h, ... \sim \pi_h, \mathcal{E}(\tilde{\pi_l})}\Bigg[\sum_{t=0,k,2k,...} \gamma_h^{t/k} A_{\pi_h}(s_t^h, a_t^h)\Bigg].
  \label{eq:new_environment_advantage}
\end{align}
等式\eqref{eq:new_environment_advantage}右边第二项的含义就是,新的整体策略相对于旧的整体策略的优势。我们想要最大化的就是这个优势。

接下来,我们定义折扣访问频率$\rho$为
\begin{align}
  \rho_{\pi_h}(s^h) = \sum_{t = 0, k, 2k, ...}\gamma_h^{t/k}P(s_t^h = s|\mathcal{E}(\pi_l))
\end{align}
当底层策略发生改变而上层策略不变的时候,$\rho_{\pi_h}$相应地变为$\tilde{\rho}_{\pi_h}$。

表达式\eqref{eq:new_environment_advantage}中,等式右侧第二项可以变形为(参见TRPO论文中的等式(2)推导,写在附录中证明2)
\begin{align}
  \sum_{s^h}\tilde{\rho}_{\pi_h}(s^h)\sum_{a^h}\pi_h(a^h|s^h)A_{\pi_h}(s^h,a^h).
  \label{eq:accurate_objective}
\end{align}
进一步,由于$s^h \sim \tilde{\rho}_{\pi_h}(s^h)$难以采样,我们采用$s^h \sim \rho_{\pi_h}(s^h)$近似,从而定义出一个需要最大化的函数,为
\begin{align}
  \sum_{s^h}\rho_{\pi_h}(s^h)\sum_{a^h}\pi_h(a^h|s^h)A_{\pi_h}(s^h,a^h).
  \label{eq:surrogate_function_for_new_environment}
\end{align}
注意到在这里,我们采用$s^h \sim \rho_{\pi_h}(s^h)$来近似$s^h \sim \tilde{\rho}_{\pi_h}(s^h)$,与TRPO采用$s^h \sim \rho_{\pi_h}(s^h)$近似$s^h \sim \rho_{\tilde{\pi}_h}(s^h)$是不同的。是否可以证明\eqref{eq:surrogate_function_for_new_environment}与表达式\eqref{eq:accurate_objective}在$\pi_h$处的一阶泰勒展开相等?这里不会证!!如果这样,只要我们优化\eqref{eq:surrogate_function_for_new_environment}的步长足够小,就能够确保\eqref{eq:accurate_objective}也变大了。(此处还有一个问题:无法证明我们的step size可以保证单调性。因为底层的TRPO算法做的事情是让\eqref{eq:TRPO_high_level}单调变大,但是并没有限制它能变大多少。也就是说,并没有满足KL constraint。也就是说,底层的TRPO最大化的objective,不等效于上层的TRPO应该最大化的objective(少了一个constraint)。这就导致我们无法证明底层的TRPO能够让上层的结果也单调递增)。

这个替代函数的形式与TRPO中提出的替代函数形式类似,只不过在这里我们是利用旧环境的采样来替代新环境的采样,而TRPO是利用旧策略的采样来替代新策略的采样。我们可以用与TRPO相同的方法把式\eqref{eq:surrogate_function_for_new_environment}变为期望的形式,注意到与$\mathcal{E}(\pi_{\theta_{old}^l})$相关其实就是与$\pi_{\theta_{old}^l}$相关,因此我们简化表达式,略去$\mathcal{E}$,同时我们引入$\pi(a|s)$的参数$\theta$,得到
\begin{align}
  \mathbb{E}_{s^h \sim \rho_{\pi_h}, a^h \sim \pi_h }[\frac{\pi_\theta^h (a^h|s^h)}{\pi_{\theta_{old}}^h(a^h|s^h)}A_{\theta_{old}}^h(s^h, a^h)].
  \label{eq:TRPO_high_level}
\end{align}
在我们实际的算法中,状态$s$可以直接来自于采样。这是因为,当我们的训练样本足够大,并且选择随机起始时,可以认为$P(s_t = s)$对于不同的$t$是相同的值,因此,训练样本中的$s$也服从$\rho_{\pi_h}(s^h)$的分布。
\begin{align}
  \mathbb{E}_{s^h, a^h \sim \pi_{\theta_{old}^h},\pi_{\theta_{old}^l}} [\frac{\pi_\theta^h (a^h|s^h)}{\pi_{\theta_{old}}^h(a^h|s^h)}A_{\theta_{old}}^h(s^h, a^h)].
  \label{eq:TRPO_high_level}
\end{align}
注意到,由式\eqref{eq:R_expectation_is_advantage},在$\pi_h$不变的条件下,我们针对底层策略$\pi^l$的更新,增大了$\mathbb{E}_{s^h, a^h \sim \pi_l, \pi_h}[A^h(s_t^h, a_t^h)]$,却没有影响$\frac{\pi_\theta^h (a^h|s^h)}{\pi_{\theta_{old}}^h(a^h|s^h)}$这一项(由于$\theta^h = \theta_{old}^h$,因此该项等于1)。这样,针对底层的更新确实使得表达式\eqref{eq:TRPO_high_level}变大了。但是这个表达式并不等于加了KL-constraint的替代函数,因此它的变大,并不能保证单调性。但在实际中,我们可以忽略掉这个限制。因此认为,底层策略的优化,就优化了整体策略度量$\eta(\pi_h)$的替代函数,从而优化了整体策略。

%%% 其它部分
\backmatter

%% 本科生要这几个索引,研究生不要。选择性留下。
% 插图索引
%\listoffigures
%% 表格索引
%\listoftables
%% 公式索引
%\listofequations


%% 参考文献
% 注意:至少需要引用一篇参考文献,否则下面两行可能引起编译错误。
% 如果不需要参考文献,请将下面两行删除或注释掉。
% 数字式引用
\bibliographystyle{thuthesis-numeric}
% 作者-年份式引用
% \bibliographystyle{thuthesis-author-year}
\bibliography{../../references/bibtex/all_references}


%% 致谢
%% 如果使用声明扫描页,将可选参数指定为扫描后的 PDF 文件名,例如:
% \begin{acknowledgement}[scan-statement.pdf]
\begin{acknowledgement}
完成本毕业设计和毕业论文,是我人生中一个里程碑式的重要节点。这份工作本身,乃至我整个本科期间的成绩和收获,离不开许多人对我的帮助。

感谢父母,在我大学四年里碰到过起起伏伏,各种小挫折,他们始终在背后支持我。他们永远是我最爱的人。

感谢常州市华英文教基金会和她的老师们。我从高三毕业获得华英奖学金,大学四年每年都收到华英老师们的关心,华英奖学金对我的持续跟踪让我不断自省和进步。

感谢大学期间我遇到过的老师们。我的班主任雷丽萍老师,虽然不负责我们的课程,但是在生活和学习的各方面都给予我许多关心。从大一以来,我从雷老师那里收获到许多关于人生规划和学习发展的建议,并且在我遇到困难的时候给我许多帮助。我的毕业设计能够完成这样一份自己较为满意的工作,也离不开雷老师的提供的指导和场地。清华大学机械工程系的许多其他的老师,以及给我上过课的所有老师们,我也都非常感激。感谢他们传授我知识,并且教会我工程师和科研人员的思维方法。

在加州大学伯克利分校交换期间,我在Masayoshi Tomizuka教授的MSC实验室进行了为期半年的研修。感谢MSC实验室的老师和师兄师姐们,给了我许多成长的建议,让我明确了自己未来的研究兴趣和方向。

本毕业设计在清华大学交叉信息研究院张崇洁老师的指导下完成。我要特别感谢张老师给我加入实验室项目的机会,允许我从零开始自学强化学习,并在我写论文的阶段给我提供了大量细致入微的指导。实验室的师姐李斯源和学弟唐敏学也和我一起合作了这个项目,我要感谢他们的帮助和讨论,使得这份工作能够成为现实。本工作也受到华为公司的资金支持,在此一并感谢。

我还要感谢本科四年遇到过的几位辅导员和所有的同学、朋友们。我的三位室友和我始终保持良好的督促关系,共同进步。我的多位好朋友,和我探讨学术问题、生活问题,乃至帮我排忧解难,没有他们,我无法想象如何在竞争激烈的清华园保持乐观的心态,健康地生存和成长。

最后,要特别感谢\thuthesis\cite{thuthesis}项目,通过开源毕业论文的\LaTeX 模板,帮助我节省了很多排版时间并且写出了一份美观的论文。

\end{acknowledgement}


%% 附录
\begin{appendix}

证明1源自TRPO论文的引理1,在表达式上有区别,但是思路相同。

\textbf{证明1:}
\begin{align}
  \tilde{\eta}(\pi_h) = \eta(\pi_h) + \mathbb{E}_{s_0^h, a_0^h, ... \sim \pi_h, \mathcal{E}(\tilde{\pi_l})}\Bigg[\sum_{t=0,k,2k,...} \gamma_h^{t/k} A_{\pi_h}(s_t^h, a_t^h)\Bigg].
  \label{eq:appendix1}
\end{align}

\textbf{证:}
\begin{align}
  &\mathbb{E}_{s_0^h, a_0^h, ... \sim \pi_h, \mathcal{E}(\tilde{\pi_l})}\Bigg[\sum_{t=0,k,2k,...} \gamma_h^{t/k} A_{\pi_h}(s_t^h, a_t^h)\Bigg]\\
  &= \mathbb{E}_{s_0^h, a_0^h, ... \sim \pi_h, \mathcal{E}(\tilde{\pi_l})}\Bigg[
      \sum_{t = 0, k, 2k, ...}\gamma_h^{t/k}(r_h(s_t^h)+\gamma_h V_{\pi_h}(s_{t+k}^h) - V_{\pi_h}(s_t^h))
      \Bigg]\\
  &= \mathbb{E}_{s_0^h, a_0^h, ... \sim \pi_h, \mathcal{E}(\tilde{\pi_l})}\Bigg[
      -V_{\pi_h}(s_0^h) + \sum_{t=0,k,2k,...} \gamma_h^{t/k} r_h(s_t^h)
      \Bigg]\\
  &= -\eta(\pi_h) + \tilde{\eta}(\pi_h)
\end{align}
存疑!这里的推导,前提条件是t一直走到无穷。对于我们这种episode会结束,并且sparse reward的情况,能否这么写??

\textbf{证明2:}
\begin{align}
    \mathbb{E}_{s_0^h, a_0^h, ... \sim \pi_h, \mathcal{E}(\tilde{\pi_l})}\Bigg[\sum_{t=0,k,2k,...} \gamma_h^{t/k} A_{\pi_h}(s_t^h, a_t^h)\Bigg] = \sum_{s^h}\tilde{\rho}_{\pi_h}(s^h)\sum_{a^h}\pi_h(a^h|s^h)A_{\pi_h}(s^h,a^h).
\end{align}

\textbf{证:}
\begin{align}
  &\mathbb{E}_{s_0^h, a_0^h, ... \sim \pi_h, \mathcal{E}(\tilde{\pi_l})}\Bigg[\sum_{t=0,k,2k,...} \gamma_h^{t/k} A_{\pi_h}(s_t^h, a_t^h)\Bigg]\\
  &= \sum_{t = 0,k,2k,...} \sum_{s^h} P(s_t^h = s^h| \pi_h, \mathcal{E}(\pi_l)) \sum_{a^h} \pi_h(a^h|s^h) \gamma_h^{t/k} A_{\pi_h}(s^h, a^h)\\
  &= \sum_{s^h} \sum_{t = 0,k,2k,...} \gamma_h^{t/k} P(s_t^h = s^h|\pi_h, \mathcal{E}(\pi_l)) \sum_{a^h}\pi_h(a^h|s^h)A_{\pi_h}(s^h,a^h)\\
  &= \sum_{s^h}\tilde{\rho}_{\pi_h}(s^h)\sum_{a^h}\pi_h(a^h|s^h)A_{\pi_h}(s^h,a^h).
\end{align}

%\input{data/appendix01}
%\include{proposal/main_for_proposal}
%\includepdf[pages=-]{proposal/main_for_proposal.pdf}

\end{appendix}

%% 个人简历
%\include{data/resume}

%% 本科生进行格式审查是需要下面这个表格,答辩可能不需要。选择性留下。
% 综合论文训练记录表
%\includepdf[pages=-]{scan-record.pdf}

\end{document}

%\renewcommand\thesection{\arabic {section}} % use this command to disable chapter numbering

\chapter{基准环境实验}
\section{基准环境介绍}
%强化学习社区统一的一套实验方案
\section{实验环境搭建}
%介绍几个环境的共同点,特点,与之前的算法比有什么区别
\section{实验结果及对比分析}
%视频实验结果截图;附上视频链接
\section{模型简化测试——控制变量}
\section{实验附录:参数和细节}
% ablation studies

\chapter{自动驾驶环境算法迁移}
%一个特殊的大实验
\section{自动驾驶实验环境架构}
\section{实验参数}
\section{实验结果及分析}
\section{自动驾驶实验总结}

\chapter{总结及结论}
\section{理论成果总结}
\section{工作内容总结}
%代码公开在Github上

%%% 其它部分
\backmatter

%% 本科生要这几个索引,研究生不要。选择性留下。
% 插图索引
\listoffigures
%% 表格索引
\listoftables
%% 公式索引
\listofequations


%% 参考文献
% 注意:至少需要引用一篇参考文献,否则下面两行可能引起编译错误。
% 如果不需要参考文献,请将下面两行删除或注释掉。
% 数字式引用
\bibliographystyle{thuthesis-numeric}
% 作者-年份式引用
% \bibliographystyle{thuthesis-author-year}
\bibliography{../../references/bibtex/all_references}


%% 致谢
% 如果使用声明扫描页,将可选参数指定为扫描后的 PDF 文件名,例如:
% \begin{acknowledgement}[scan-statement.pdf]
\begin{acknowledgement}
完成本毕业设计和毕业论文,是我人生中一个里程碑式的重要节点。这份工作本身,乃至我整个本科期间的成绩和收获,离不开许多人对我的帮助。

感谢父母,在我大学四年里碰到过起起伏伏,各种小挫折,他们始终在背后支持我。他们永远是我最爱的人。

感谢常州市华英文教基金会和她的老师们。我从高三毕业获得华英奖学金,大学四年每年都收到华英老师们的关心,华英奖学金对我的持续跟踪让我不断自省和进步。

感谢大学期间我遇到过的老师们。我的班主任雷丽萍老师,虽然不负责我们的课程,但是在生活和学习的各方面都给予我许多关心。从大一以来,我从雷老师那里收获到许多关于人生规划和学习发展的建议,并且在我遇到困难的时候给我许多帮助。我的毕业设计能够完成这样一份自己较为满意的工作,也离不开雷老师的提供的指导和场地。清华大学机械工程系的许多其他的老师,以及给我上过课的所有老师们,我也都非常感激。感谢他们传授我知识,并且教会我工程师和科研人员的思维方法。

在加州大学伯克利分校交换期间,我在Masayoshi Tomizuka教授的MSC实验室进行了为期半年的研修。感谢MSC实验室的老师和师兄师姐们,给了我许多成长的建议,让我明确了自己未来的研究兴趣和方向。

本毕业设计在清华大学交叉信息研究院张崇洁老师的指导下完成。我要特别感谢张老师给我加入实验室项目的机会,允许我从零开始自学强化学习,并在我写论文的阶段给我提供了大量细致入微的指导。实验室的师姐李斯源和学弟唐敏学也和我一起合作了这个项目,我要感谢他们的帮助和讨论,使得这份工作能够成为现实。本工作也受到华为公司的资金支持,在此一并感谢。

我还要感谢本科四年遇到过的几位辅导员和所有的同学、朋友们。我的三位室友和我始终保持良好的督促关系,共同进步。我的多位好朋友,和我探讨学术问题、生活问题,乃至帮我排忧解难,没有他们,我无法想象如何在竞争激烈的清华园保持乐观的心态,健康地生存和成长。

最后,要特别感谢\thuthesis\cite{thuthesis}项目,通过开源毕业论文的\LaTeX 模板,帮助我节省了很多排版时间并且写出了一份美观的论文。

\end{acknowledgement}


%% 附录
\begin{appendix}
%\input{data/appendix01}
%\include{proposal/main_for_proposal}
\includepdf[pages=-]{proposal/main_for_proposal.pdf}
\end{appendix}

%% 个人简历
%\include{data/resume}

%% 本科生进行格式审查是需要下面这个表格,答辩可能不需要。选择性留下。
% 综合论文训练记录表
\includepdf[pages=-]{scan-record.pdf}

\end{document}
