\documentclass[degree=bachelor, tocarialchapter, pifootnote]{thuthesis}
% 选项
%   degree=[bachelor|master|doctor|postdoctor], % 必选,学位类型
%   secret,                % 可选(默认:关闭),是否有密级
%   tocarialchapter,       % 可选(默认:关闭),章目录中使用黑体(这项表示同时打开下面两项)
%   tocarialchapterentry,  % 可选(默认:关闭),单独控制章标题在目录中使用黑体
%   tocarialchapterpage,   % 可选(默认:关闭),单独控制章页码在目录中使用黑体
%   pifootnote,            % 可选(默认:关闭),页脚编号采用 pifont 字体符号,建议打开

% 所有其它可能用到的包都统一放到这里了,可以根据自己的实际添加或者删除。
\usepackage{thuthesis}

% 下面的包让argmax之类的,_{}的东西显示在argmax正下方
\usepackage{amsmath}
\DeclareMathOperator*{\argmax}{argmax}
\DeclareMathOperator*{\argmin}{argmin}

% 定义所有的图片文件在 figures 子目录下
\graphicspath{{figures/}}

% 可以在这里修改配置文件中的定义。导言区可以使用中文。
\def\myname{王芮}

\begin{document}

%%% 封面部分å
%\frontmatter

%\thusetup{
  %******************************
  % 注意:
  %   1. 配置里面不要出现空行
  %   2. 不需要的配置信息可以删除
  %******************************
  %
  %=====
  % 秘级
  %=====
%  secretlevel={秘密},
%  secretyear={10},
  %
  %=========
  % 中文信息
  %=========
  %ctitle={分层强化学习算法及其在机器人仿真环境中的应用研究},
  ctitle={HAAR:基于优势函数辅助奖励的分层强化学习算法},
  cdegree={工学学士},
  cdepartment={机械工程系},
  cmajor={机械工程},
  cauthor={王芮},
  csupervisor={雷丽萍副教授},
%  ccosupervisor={某某某教授}, % 联合指导老师
  % 日期自动使用当前时间,若需指定按如下方式修改:
  % cdate={超新星纪元},
  %
  % 博士后专有部分
%  cfirstdiscipline={计算机科学与技术},
%  cseconddiscipline={系统结构},
%  postdoctordate={2009年7月——2011年7月},
%  id={编号}, % 可以留空: id={},
%  udc={UDC}, % 可以留空
%  catalognumber={分类号}, % 可以留空
  %
  %=========
  % 英文信息
  %=========
%  etitle={An Introduction to \LaTeX{} Thesis Template of Tsinghua University v\version},
  % 这块比较复杂,需要分情况讨论:
  % 1. 学术型硕士
  %    edegree:必须为Master of Arts或Master of Science(注意大小写)
  %             “哲学、文学、历史学、法学、教育学、艺术学门类,公共管理学科
  %              填写Master of Arts,其它填写Master of Science”
  %    emajor:“获得一级学科授权的学科填写一级学科名称,其它填写二级学科名称”
  % 2. 专业型硕士
  %    edegree:“填写专业学位英文名称全称”
  %    emajor:“工程硕士填写工程领域,其它专业学位不填写此项”
  % 3. 学术型博士
  %    edegree:Doctor of Philosophy(注意大小写)
  %    emajor:“获得一级学科授权的学科填写一级学科名称,其它填写二级学科名称”
  % 4. 专业型博士
  %    edegree:“填写专业学位英文名称全称”
  %    emajor:不填写此项
%  edegree={Doctor of Engineering},
%  emajor={Computer Science and Technology},
%  eauthor={Xue Ruini},
%  esupervisor={Professor Zheng Weimin},
%  eassosupervisor={Chen Wenguang},
  % 日期自动生成,若需指定按如下方式修改:
  % edate={December, 2005}
  %
  % 关键词用“英文逗号”分割
  ckeywords={强化学习, 分层强化学习, 机器人学, 规划算法, 自动驾驶, 路径规划, 优化, 策略单调性},
  ekeywords={Reinforcement Learning, Hierarchical Reinforcement Learning, Robotics, Planning Algorithms, Self-Driving, Path Planning, Optimization, Monotonic Improvement}
}

% 定义中英文摘要和关键字
\begin{cabstract}
强化学习问题中的一个重要挑战是环境的稀疏奖励问题。如果环境反馈给智能体的信号过于稀疏,智能体的学习速度会非常缓慢。为此,人们提出分层强化学习的方法,来帮助强化学习算法提高采样效率,加速学习。在分层强化学习范式当中,特别地,我们需要为下层策略设计一个辅助奖励函数,从而促进下层策略的学习。在本篇论文中,我们提出了一个分层强化学习的框架,通过基于优势函数的下层奖励函数设计,它可以单调地同时优化上下层策略。截止目前,绝大多数最前沿的分层强化学习算法都对状态空间的表达非常敏感,或者依赖于手动设计的、只在特定环境下可用的下层奖励函数。得益于我们的辅助奖励函数设计只与优势函数的值相关这一特点,我们的算法不依赖于状态空间的表示方式,因此和其他方法相比更为通用和实用,尤其是在机器人连续动作空间这类应用场景。我们从理论上证明了这种算法可以保留上下层各自优化算法的单调特性,使整体策略也得到单调优化。实验方面,在Mujoco仿真环境的多个任务中,我们的算法表现均大大优于现有的最佳算法。最后,我们还搭建了一个自动驾驶仿真环境来验证本算法在更现实和复杂的环境下的表现。我们设计了迁移实验,将算法从Mujoco实验环境迁移到了自动驾驶环岛路径规划问题中,同时也指出了在这个环境中的下一步研究方向。
\end{cabstract}

\begin{eabstract}
One of the key challenges in reinforcement learning is related to sparse reward signals from the environment. Hierarchical reinforcement learning (HRL) can be utilized to solve the credit assignment problem in a sparse reward task. In the HRL paradigms, we need to design a low-level auxiliary reward function to facilitate low-level learning. In this paper, we propose an HRL framework which monotonically  optimizes the joint policy of the upper-level and lower-level  through the design of an advantage function-based auxiliary reward. Most state-of-the-art HRL algorithms are sensitive to state representations, or require careful design of the auxiliary reward functions  for low-level training. Our method is more generic and applicable in continuous control tasks since our auxiliary reward for low-level policy update is dependent only on the advantage function of the high-level policy, therefore invariant of the state representations. We theoretically prove that monotonic improvement of the joint-policy is guaranteed in our method. Experimental results show that our algorithm performs dramatically better than other state-of-the-art methods across multiple tasks in the Mujoco learning environment. Finally, we design a complex autonomous-driving simulation environment as a testbed for our algorithm in more realistic settings. We transfer our algorithm to this new environment and report the experiment results of our framework on the self-driving car's planning task at a roundabout, as well as point out the direction for future research.

\end{eabstract}

% 如果习惯关键字跟在摘要文字后面,可以用直接命令来设置,如下:
% \ckeywords{\TeX, \LaTeX, CJK, 模板, 论文}

% \ekeywords{\TeX, \LaTeX, CJK, template, thesis}

% 如果使用授权说明扫描页,将可选参数中指定为扫描得到的 PDF 文件名,例如:
% \makecover[scan-auth.pdf]
%\makecover
%% 目录
% 设置目录层级,三级小标题全部显示
%\setcounter{secnumdepth}{5}
%\setcounter{tocdepth}{5}
%\tableofcontents

%% 符号对照表
%\begin{denotation}[3cm]
\item[$s$] 状态(任意状态)
\item[$a$] 行动(任意行动)
\item[$\pi$] 策略
\item[$\pi(s)$] 策略函数
\item[$\pi(a|s)$] 策略在状态$s$下采取行动$a$的概率
\item[$r$] 奖励
\item[$r(s, a)$] 在状态$s$下采取行动$a$获得的奖励
\item[$G_t$] 时间$t$之后的总奖励
\item[$v(s)$] 状态$s$的价值函数
\item[$q_\pi(s, a)$] 在策略$\pi$下,在状态$s$下采取行动$a$的价值
\item[$V(s)$] 状态$s$的价值函数的估计值
\item[$Q(s, a)$] 状态$s$,动作$a$的价值的估计值
\item[$p(s'|s, a)$] 在状态$s$采取动作$a$,状态转移到$s'$的概率
\item[$p(s', r|s, a)$] 在状态$s$采取动作$a$,状态转移到$s'$并且获得奖励$r$的概率
\item[$\pi(a|s, \theta)$] 策略函数的参数是$\theta$,在状态$s$下采取行动$a$的概率
\item[$\pi_\theta$] 策略函数,其参数为$\theta$
\item[$\bigtriangledown \pi(a|s, \theta)$] 策略函数的参数是$\theta$,在状态$s$下采取行动$a$的概率对$\theta$的导数
\item[$\gamma$] 总奖励$G_t$计算时,对于后续$R_{t+t_i}$的折扣率
\item[$\rho$] 重要性采样比例(importance-sampling ratio)
\item[$\delta_t$] 时序差异误差(TD error)
\item[RL] 强化学习(reinforcement learning)
\item[MPC] 模型预测控制(model-predictive control)
\item[TD] 时序差异(temporal-difference)
\item[MDP] 马尔可夫过程 (Markov Decision Process)
\item[POMDP] 部分可观测的马尔可夫过程 (Partially Observable Markov Decision Process)
\item[DQN] 深度Q-学习神经网络(Deep Q Network)
\item[DRQN] 深度循环Q-学习神经网络(Deep Recurrent Q Network)
\item[SNN] 随机神经网络(Stochastic Neural Network)
\end{denotation}



%%% 正文部分
\mainmatter
%\include{data/chap01}
%\include{data/chap02}
%\renewcommand\thesection{\arabic {section}} % use this command to disable chapter numbering

\section{Derivation}
我们的算法是分别对上下层策略进行更新,接下来我们证明,单独更新上层策略和单独更新下层策略都可以使得整体策略变优。

在接下来的推导过程中,我们用上标或下标$h$和$l$来表示强化学习中的一系列参数,是针对上层策略(high)还是下层策略(low)。

首先,我们证明,在下层策略不变的情况下,利用TRPO更新上层策略,可以保证整体策略的单调递增,即$V_h(s_0^h)$单调递增。这一步很简单,既然底层策略不变,那么可以认为底层策略是包含在环境中的,因此TRPO去优化上层策略,就相当于TRPO直接作用于一个单层的强化学习结构,对应的优化目标必然是越来越好的。注意到对于上层策略来说,它得到的$R$就是上下层策略作为一个整体得到的真实$R$。

接下来,我们推导,采用$ \frac{\gamma_h V_h(s_{t + k}^h) - V_h(s_{t}^h)}{k}$作为上层给下层的输入,可以使得整体策略变优。

我们的上下层优化算法采用的均是TRPO\cite{TRPO}。在TRPO当中,我们最终希望去优化的东西是$V_{\pi}(s_0)$。TRPO算法等价于在基于策略梯度定理给出的策略梯度优化中,增加了对步长的限制,从而确保策略更新的单调性。在策略梯度方法的论文\cite{policy_gradient_theorem}中,Sutton指出,对一个策略优化的目标函数$J$既可以采用$V_{\pi}(s_0)$的定义,也可以采用$R$的平均值的定义,它们可以分别推导得出策略梯度定理,而这个定理中的策略梯度,与TRPO试图优化的替代函数的梯度是相等的。因此,TRPO也可以理解成是在最大化$\mathbb{E}_{s, a \sim \pi}[R(s, a)]$。

如果把$R$的均值作为优化目标,我们有
\begin{align}
  \bigtriangledown J(\theta) = \bigtriangledown \mathbb{E}_{s\sim\pi}[\sum_{a} \pi(s, a) R(s, a)]
  &= \bigtriangledown \mathbb{E}_{s, a\sim\pi}[R(s, a)].
  \label{eq:policy_gradient}
\end{align}

我们定义了
\begin{align}
  R_l(s_{t + i}^l, a_{t + i}^l)|_{i = 0,1,...,k-1} = \frac{V_h(s_{t + k}^h) - V_h(s_{t}^h)}{k}.
  \label{eq:R_low}
\end{align}

前面我们提到,TRPO完成的任务是最大化$\mathbb{E}_{s, a \sim \pi}[R(s, a)]$,因此我们在利用TRPO针对底层策略进行优化的时候,相当于最大化$\mathbb{E}_{s, a \sim \pi_l, \pi_h}[R_l(s_l, a_l)]$。注意这里,我们把$\pi_h$看做是一个固定的概率分布函数,而不对它进行优化。$\pi_h$可以看做是环境的一部分,而它输出的隐式编码(latent code)则是我们观察(observation)的一部分。

根据\eqref{eq:R_low}中的结果,我们两边取期望,可以得到
\begin{align}
  \mathbb{E}_{s^l, a^l \sim \pi_l, \pi_h}[R_l(s^l, a^l)] = \frac{1}{k} \mathbb{E}_{s^h, a^h \sim \pi_l, \pi_h}[\gamma_h V_h(s_{t + k}^h) - V_h(s_{t}^h)].
  \label{eq:R_expectation}
\end{align}

回忆在强化学习中,我们关于优势函数$A(s, a)$的定义,为
\begin{align}
  A(S_t, A_t) = Q(S_t, A_t) - V(S_t) = R(S_t, A_t) + \gamma V(S_{t + 1}) - V(s).
  \label{eq:advantage}
\end{align}

由于在稀疏强化学习问题中有稀疏奖励条件:
\begin{align}
  R(S_t, A_t) = 0, \forall t \neq t_{end}.
  \label{eq:sparse_reward_condition}
\end{align}

因此,\eqref{eq:advantage}变为
\begin{align}
  A(S_t, A_t) = \gamma V(S_{t + 1}) - V(s).
  \label{eq:sparse_advantage}
\end{align}

由\eqref{eq:R_expectation}和\eqref{eq:sparse_advantage}可知,TRPO优化下层策略的结果,等效于优化了上层的优势函数,也就是
\begin{align}
  \mathbb{E}_{s^l, a^l \sim \pi_l, \pi_h}[R_l(s^l, a^l)] = \frac{1}{k} \mathbb{E}_{s^h, a^h \sim \pi_l, \pi_h}[A^h(s_t^h, a_t^h)].
  \label{eq:R_expectation_is_advantage}
\end{align}

接下来我们证明,通过优化这个期望值,就近似等效于优化了整体策略的表现。这里不能直接把TRPO的(13)搬过来用在上层policy上!TRPO是建立在环境动态不变的基础上的,但是我们的环境变了。必须follow TRPO论文的推导,看一下我们能不能推出一样的结果。下面做的就是这个事情。
%我们已经知道,TRPO最大化的替代函数的梯度,与策略梯度定理指出的梯度成正比。因而这个梯度,对应的需要最大化的函数,可以和TRPO(应用于上层策略)给出的替代函数相同。接下来我们推导在下层策略可能发生改变的情况下,上层策略的目标函数(也就是整体策略的目标函数)如何变化。

注意到,对于上层策略来说,底层策略相当于环境动力学(dynamics)。当底层策略发生了变化时,可以看做是环境动力学发生了变化。因此,即使上层策略不变,$S^h_t$的分布也可能发生改变。我们用$\mathcal{E}(\pi_l)$表示环境,并且环境与$\pi_l$有关。定义$\pi_h$带来的折扣奖励期望
\begin{equation}
\begin{aligned}
  \eta(\pi_h) = \mathbb{E}_{s_0^h, a_0^h, ...}\Bigg[\sum_{t = 0, k, 2k, ...} \gamma_h^{t/k} r_h(s_t^h, a_t^h)\Bigg] \text{ ,  where}   \\
  s_0^h \sim \rho_0^h(s_0^h), a_t^h \sim \pi_h(s_t^h), s_{t+1} \sim P(s_{t+1}^h|s_t^h, a_t^h, \mathcal{E}(\pi_l))
\end{aligned}
\end{equation}

我们用$\tilde{\eta}(\pi_h)$来表示$\pi_l$发生变化(变为$\tilde{\pi_l}$)而$\pi_h$没有发生变化以后的折扣奖励值期望。将TRPO根据\cite{TRPO_pre}得到的引理1(\textbf{Lemma 1})稍加变形,我们可以得到类似的结果:(具体推导参见附录中证明1)
\begin{align}
  \tilde{\eta}(\pi_h) = \eta(\pi_h) + \mathbb{E}_{s_0^h, a_0^h, ... \sim \pi_h, \mathcal{E}(\tilde{\pi_l})}\Bigg[\sum_{t=0,k,2k,...} \gamma_h^{t/k} A_{\pi_h}(s_t^h, a_t^h)\Bigg].
  \label{eq:new_environment_advantage}
\end{align}
等式\eqref{eq:new_environment_advantage}右边第二项的含义就是,新的整体策略相对于旧的整体策略的优势。我们想要最大化的就是这个优势。

接下来,我们定义折扣访问频率$\rho$为
\begin{align}
  \rho_{\pi_h}(s^h) = \sum_{t = 0, k, 2k, ...}\gamma_h^{t/k}P(s_t^h = s|\mathcal{E}(\pi_l))
\end{align}
当底层策略发生改变而上层策略不变的时候,$\rho_{\pi_h}$相应地变为$\tilde{\rho}_{\pi_h}$。

表达式\eqref{eq:new_environment_advantage}中,等式右侧第二项可以变形为(参见TRPO论文中的等式(2)推导,写在附录中证明2)
\begin{align}
  \sum_{s^h}\tilde{\rho}_{\pi_h}(s^h)\sum_{a^h}\pi_h(a^h|s^h)A_{\pi_h}(s^h,a^h).
  \label{eq:accurate_objective}
\end{align}
进一步,由于$s^h \sim \tilde{\rho}_{\pi_h}(s^h)$难以采样,我们采用$s^h \sim \rho_{\pi_h}(s^h)$近似,从而定义出一个需要最大化的函数,为
\begin{align}
  \sum_{s^h}\rho_{\pi_h}(s^h)\sum_{a^h}\pi_h(a^h|s^h)A_{\pi_h}(s^h,a^h).
  \label{eq:surrogate_function_for_new_environment}
\end{align}
注意到在这里,我们采用$s^h \sim \rho_{\pi_h}(s^h)$来近似$s^h \sim \tilde{\rho}_{\pi_h}(s^h)$,与TRPO采用$s^h \sim \rho_{\pi_h}(s^h)$近似$s^h \sim \rho_{\tilde{\pi}_h}(s^h)$是不同的。是否可以证明\eqref{eq:surrogate_function_for_new_environment}与表达式\eqref{eq:accurate_objective}在$\pi_h$处的一阶泰勒展开相等?这里不会证!!如果这样,只要我们优化\eqref{eq:surrogate_function_for_new_environment}的步长足够小,就能够确保\eqref{eq:accurate_objective}也变大了。(此处还有一个问题:无法证明我们的step size可以保证单调性。因为底层的TRPO算法做的事情是让\eqref{eq:TRPO_high_level}单调变大,但是并没有限制它能变大多少。也就是说,并没有满足KL constraint。也就是说,底层的TRPO最大化的objective,不等效于上层的TRPO应该最大化的objective(少了一个constraint)。这就导致我们无法证明底层的TRPO能够让上层的结果也单调递增)。

这个替代函数的形式与TRPO中提出的替代函数形式类似,只不过在这里我们是利用旧环境的采样来替代新环境的采样,而TRPO是利用旧策略的采样来替代新策略的采样。我们可以用与TRPO相同的方法把式\eqref{eq:surrogate_function_for_new_environment}变为期望的形式,注意到与$\mathcal{E}(\pi_{\theta_{old}^l})$相关其实就是与$\pi_{\theta_{old}^l}$相关,因此我们简化表达式,略去$\mathcal{E}$,同时我们引入$\pi(a|s)$的参数$\theta$,得到
\begin{align}
  \mathbb{E}_{s^h \sim \rho_{\pi_h}, a^h \sim \pi_h }[\frac{\pi_\theta^h (a^h|s^h)}{\pi_{\theta_{old}}^h(a^h|s^h)}A_{\theta_{old}}^h(s^h, a^h)].
  \label{eq:TRPO_high_level}
\end{align}
在我们实际的算法中,状态$s$可以直接来自于采样。这是因为,当我们的训练样本足够大,并且选择随机起始时,可以认为$P(s_t = s)$对于不同的$t$是相同的值,因此,训练样本中的$s$也服从$\rho_{\pi_h}(s^h)$的分布。
\begin{align}
  \mathbb{E}_{s^h, a^h \sim \pi_{\theta_{old}^h},\pi_{\theta_{old}^l}} [\frac{\pi_\theta^h (a^h|s^h)}{\pi_{\theta_{old}}^h(a^h|s^h)}A_{\theta_{old}}^h(s^h, a^h)].
  \label{eq:TRPO_high_level}
\end{align}
注意到,由式\eqref{eq:R_expectation_is_advantage},在$\pi_h$不变的条件下,我们针对底层策略$\pi^l$的更新,增大了$\mathbb{E}_{s^h, a^h \sim \pi_l, \pi_h}[A^h(s_t^h, a_t^h)]$,却没有影响$\frac{\pi_\theta^h (a^h|s^h)}{\pi_{\theta_{old}}^h(a^h|s^h)}$这一项(由于$\theta^h = \theta_{old}^h$,因此该项等于1)。这样,针对底层的更新确实使得表达式\eqref{eq:TRPO_high_level}变大了。但是这个表达式并不等于加了KL-constraint的替代函数,因此它的变大,并不能保证单调性。但在实际中,我们可以忽略掉这个限制。因此认为,底层策略的优化,就优化了整体策略度量$\eta(\pi_h)$的替代函数,从而优化了整体策略。

%%% 其它部分
\backmatter

%% 本科生要这几个索引,研究生不要。选择性留下。
% 插图索引
%\listoffigures
%% 表格索引
%\listoftables
%% 公式索引
%\listofequations


%% 参考文献
% 注意:至少需要引用一篇参考文献,否则下面两行可能引起编译错误。
% 如果不需要参考文献,请将下面两行删除或注释掉。
% 数字式引用
\bibliographystyle{thuthesis-numeric}
% 作者-年份式引用
% \bibliographystyle{thuthesis-author-year}
\bibliography{../../references/bibtex/all_references}


%% 致谢
%% 如果使用声明扫描页,将可选参数指定为扫描后的 PDF 文件名,例如:
% \begin{acknowledgement}[scan-statement.pdf]
\begin{acknowledgement}
完成本毕业设计和毕业论文,是我人生中一个里程碑式的重要节点。这份工作本身,乃至我整个本科期间的成绩和收获,离不开许多人对我的帮助。

感谢父母,在我大学四年里碰到过起起伏伏,各种小挫折,他们始终在背后支持我。他们永远是我最爱的人。

感谢常州市华英文教基金会和她的老师们。我从高三毕业获得华英奖学金,大学四年每年都收到华英老师们的关心,华英奖学金对我的持续跟踪让我不断自省和进步。

感谢大学期间我遇到过的老师们。我的班主任雷丽萍老师,虽然不负责我们的课程,但是在生活和学习的各方面都给予我许多关心。从大一以来,我从雷老师那里收获到许多关于人生规划和学习发展的建议,并且在我遇到困难的时候给我许多帮助。我的毕业设计能够完成这样一份自己较为满意的工作,也离不开雷老师的提供的指导和场地。清华大学机械工程系的许多其他的老师,以及给我上过课的所有老师们,我也都非常感激。感谢他们传授我知识,并且教会我工程师和科研人员的思维方法。

在加州大学伯克利分校交换期间,我在Masayoshi Tomizuka教授的MSC实验室进行了为期半年的研修。感谢MSC实验室的老师和师兄师姐们,给了我许多成长的建议,让我明确了自己未来的研究兴趣和方向。

本毕业设计在清华大学交叉信息研究院张崇洁老师的指导下完成。我要特别感谢张老师给我加入实验室项目的机会,允许我从零开始自学强化学习,并在我写论文的阶段给我提供了大量细致入微的指导。实验室的师姐李斯源和学弟唐敏学也和我一起合作了这个项目,我要感谢他们的帮助和讨论,使得这份工作能够成为现实。本工作也受到华为公司的资金支持,在此一并感谢。

我还要感谢本科四年遇到过的几位辅导员和所有的同学、朋友们。我的三位室友和我始终保持良好的督促关系,共同进步。我的多位好朋友,和我探讨学术问题、生活问题,乃至帮我排忧解难,没有他们,我无法想象如何在竞争激烈的清华园保持乐观的心态,健康地生存和成长。

最后,要特别感谢\thuthesis\cite{thuthesis}项目,通过开源毕业论文的\LaTeX 模板,帮助我节省了很多排版时间并且写出了一份美观的论文。

\end{acknowledgement}


%% 附录
\begin{appendix}

证明1源自TRPO论文的引理1,在表达式上有区别,但是思路相同。

\textbf{证明1:}
\begin{align}
  \tilde{\eta}(\pi_h) = \eta(\pi_h) + \mathbb{E}_{s_0^h, a_0^h, ... \sim \pi_h, \mathcal{E}(\tilde{\pi_l})}\Bigg[\sum_{t=0,k,2k,...} \gamma_h^{t/k} A_{\pi_h}(s_t^h, a_t^h)\Bigg].
  \label{eq:appendix1}
\end{align}

\textbf{证:}
\begin{align}
  &\mathbb{E}_{s_0^h, a_0^h, ... \sim \pi_h, \mathcal{E}(\tilde{\pi_l})}\Bigg[\sum_{t=0,k,2k,...} \gamma_h^{t/k} A_{\pi_h}(s_t^h, a_t^h)\Bigg]\\
  &= \mathbb{E}_{s_0^h, a_0^h, ... \sim \pi_h, \mathcal{E}(\tilde{\pi_l})}\Bigg[
      \sum_{t = 0, k, 2k, ...}\gamma_h^{t/k}(r_h(s_t^h)+\gamma_h V_{\pi_h}(s_{t+k}^h) - V_{\pi_h}(s_t^h))
      \Bigg]\\
  &= \mathbb{E}_{s_0^h, a_0^h, ... \sim \pi_h, \mathcal{E}(\tilde{\pi_l})}\Bigg[
      -V_{\pi_h}(s_0^h) + \sum_{t=0,k,2k,...} \gamma_h^{t/k} r_h(s_t^h)
      \Bigg]\\
  &= -\eta(\pi_h) + \tilde{\eta}(\pi_h)
\end{align}
存疑!这里的推导,前提条件是t一直走到无穷。对于我们这种episode会结束,并且sparse reward的情况,能否这么写??

\textbf{证明2:}
\begin{align}
    \mathbb{E}_{s_0^h, a_0^h, ... \sim \pi_h, \mathcal{E}(\tilde{\pi_l})}\Bigg[\sum_{t=0,k,2k,...} \gamma_h^{t/k} A_{\pi_h}(s_t^h, a_t^h)\Bigg] = \sum_{s^h}\tilde{\rho}_{\pi_h}(s^h)\sum_{a^h}\pi_h(a^h|s^h)A_{\pi_h}(s^h,a^h).
\end{align}

\textbf{证:}
\begin{align}
  &\mathbb{E}_{s_0^h, a_0^h, ... \sim \pi_h, \mathcal{E}(\tilde{\pi_l})}\Bigg[\sum_{t=0,k,2k,...} \gamma_h^{t/k} A_{\pi_h}(s_t^h, a_t^h)\Bigg]\\
  &= \sum_{t = 0,k,2k,...} \sum_{s^h} P(s_t^h = s^h| \pi_h, \mathcal{E}(\pi_l)) \sum_{a^h} \pi_h(a^h|s^h) \gamma_h^{t/k} A_{\pi_h}(s^h, a^h)\\
  &= \sum_{s^h} \sum_{t = 0,k,2k,...} \gamma_h^{t/k} P(s_t^h = s^h|\pi_h, \mathcal{E}(\pi_l)) \sum_{a^h}\pi_h(a^h|s^h)A_{\pi_h}(s^h,a^h)\\
  &= \sum_{s^h}\tilde{\rho}_{\pi_h}(s^h)\sum_{a^h}\pi_h(a^h|s^h)A_{\pi_h}(s^h,a^h).
\end{align}

%\input{data/appendix01}
%\include{proposal/main_for_proposal}
%\includepdf[pages=-]{proposal/main_for_proposal.pdf}

\end{appendix}

%% 个人简历
%\include{data/resume}

%% 本科生进行格式审查是需要下面这个表格,答辩可能不需要。选择性留下。
% 综合论文训练记录表
%\includepdf[pages=-]{scan-record.pdf}

\end{document}
